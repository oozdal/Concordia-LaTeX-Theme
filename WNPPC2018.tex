\documentclass[hyperref={bookmarks=false},aspectratio=169]{beamer}
\usepackage[utf8]{inputenc}
\usepackage{xparse}

\def\singR{\mathcal{X}_{R}}
\def\singRbar{\mathcal{\overline{X}}_{R}}

\newcommand*{\graybullet}{\textcolor{gray}{\textbullet}}
\newcommand*{\bluebullet}{\textcolor{blue}{\textbullet}}
\newcommand*{\redbullet}{\textcolor{red}{\textbullet}}


\NewDocumentCommand{\tens}{t_}
{%
	\IfBooleanTF{#1}
	{\tensop}
	{\otimes}%
}
\NewDocumentCommand{\tensop}{m}
{%
	\mathbin{\mathop{\otimes}\displaylimits_{#1}}%
}

% ---------------  Define theme and color scheme  -----------------
\usetheme[sidebarleft]{Caltech}  % 3 options: minimal, sidebarleft, sidebarright

%\setbeamertemplate{footline}[frame number]

% ------------  Information on the title page  --------------------
\title[]
{\bfseries{Exploring the  supersymmetric $U(1)_{B-L} \times U(1)_{R}$ model with dark matter, muon $g-2$ and $Z^\prime$ mass limits}}

\subtitle{{\small based on \hyperlink{https://journals.aps.org/prd/abstract/10.1103/PhysRevD.97.015012}{Phys. Rev. D 97, 015012}}}

\author[]
{\underline{Özer Özdal}\inst{1} \and Mariana Frank\inst{1}}

\institute[Concordia University]
{
    CONCORDIA UNIVERSITY\inst{1}
}


\date[WNPFC, 2018]
{Winter Nuclear \& Particle Physics Conference\\
	Mont Tremblant, Québec\\
	February 17, 2018}	

% logo of my university
\titlegraphic{\includegraphics[width=4cm]{./figures/concordia-logo.png}}

%------------------------------------------------------------

%------------------------------------------------------------
%The next block of commands puts the table of contents at the 
%beginning of each section and highlights the current section:

\AtBeginSection[]
{
  \begin{frame}
    \frametitle{Outline}
    \tableofcontents[currentsection]
  \end{frame}
}
%------------------------------------------------------------


\begin{document}

\frame{\titlepage}  % Creates title page

%---------   table of contents after title page  ------------
\begin{frame}
\frametitle{Outline}
\tableofcontents
\end{frame}
%---------------------------------------------------------

\section{Introduction}
\subsection{The Standard Model (SM)}

%---------------------------------------------------------

\begin{frame}
\frametitle{The Standard Model of Particle Physics}

\begin{columns}
	
	\column{0.60\textwidth}
	
	\begin{figure}
		\centering
		\includegraphics[width=\columnwidth]{./figures/SMparticlecontent.png}
	\end{figure}
	
	
	
	\column{0.40\textwidth}
	 {\small $G_{321} = SU(3)_C \tens \underbrace{SU(2)_L \tens U(1)_Y }_{\downarrow}   $}
	 
	 \pause
	 
	 \hspace{1.735cm} {\small $SU(3)_C \tens U(1)_{EM}$}
	 
\end{columns}


\end{frame}

%---------------------------------------------------------

\begin{frame}
\frametitle{The Standard Model of Particle Physics}

\begin{columns}
	
	\column{0.50\textwidth}
	
	\begin{figure}
		\centering
		\includegraphics[width=\columnwidth]{./figures/SMcontentquarkfocus.png}
	\end{figure}
	
	\column{0.50\textwidth}
	\centering
	{\small $G_{321} = SU(3)_C \tens \underbrace{SU(2)_L \tens U(1)_Y }_{\downarrow}   $}
	
	\hspace{1.735cm} {\small $SU(3)_C \tens U(1)_{EM}$}
	
	\begin{table}[]
		\centering
		{\tiny 	\begin{tabular}{c|c|c|c|c}
				&  $SU(3)_C \times SU(2)_L  \times U(1)_Y$ & $I_3$  & $Q_{\rm EM}$    \\ \hline \hline
				$Q=\begin{pmatrix} 
				u_{L}  \\
				d_{L} 
				\end{pmatrix}$	  & (\textbf{3}, \textbf{2}, $\frac{1}{3}$)  & $\begin{matrix} 
				1/2  \\
				-1/2 
				\end{matrix}$  & $\begin{matrix} 
				2/3  \\
				-1/3 
				\end{matrix}$      \\ 
				
	$u_R $ & $(\overline{\textbf{3}}, \textbf{1}, \frac{4}{3})$ & 0 & $2/3$ \\
	$d_R $ & $(\overline{\textbf{3}}, \textbf{1}, -\frac{2}{3})$ & 0 & $-1/3$ \\ \hline 
				
	   		\end{tabular}}
	\end{table}
	
	
\end{columns}

\end{frame}

%---------------------------------------------------------

\begin{frame}
\frametitle{The Standard Model of Particle Physics}

\begin{columns}
	
	\column{0.50\textwidth}
	
	\begin{figure}
		\centering
		\includegraphics[width=\columnwidth]{./figures/SMcontentleptonfocus.png}
	\end{figure}
	
	\column{0.50\textwidth}
	\centering
	{\small $G_{321} = SU(3)_C \tens \underbrace{SU(2)_L \tens U(1)_Y }_{\downarrow}   $}
	
	\hspace{1.735cm} {\small $SU(3)_C \tens U(1)_{EM}$}
	
	
		\begin{table}[]
		\centering
		{\tiny 	\begin{tabular}{c|c|c|c|c}
				&  $SU(3)_C \times SU(2)_L  \times U(1)_Y$ & $I_3$  & $Q_{\rm EM}$    \\ \hline \hline
				$Q=\begin{pmatrix} 
				u_{L}  \\
				d_{L} 
				\end{pmatrix}$	  & (\textbf{3}, \textbf{2}, $\frac{1}{3}$)  & $\begin{matrix} 
				1/2  \\
				-1/2 
				\end{matrix}$  & $\begin{matrix} 
				2/3  \\
				-1/3 
				\end{matrix}$      \\ 
				
				$u_R $ & $(\overline{\textbf{3}}, \textbf{1}, \frac{4}{3})$ & 0 & $2/3$ \\
				$d_R $ & $(\overline{\textbf{3}}, \textbf{1}, -\frac{2}{3})$ & 0 & $-1/3$ \\ \hline 
				
				
				$L=\begin{pmatrix} 
				\nu_{L}  \\
				e_{L} 
				\end{pmatrix}$	  & (\textbf{1}, \textbf{2}, $-1$)  & $\begin{matrix} 
				1/2  \\
				-1/2 
				\end{matrix}$  & $\begin{matrix} 
				0  \\
				-1 
				\end{matrix}$      \\ 
				
				$e_R $ & $(\overline{\textbf{1}}, \textbf{1}, -2)$ & 0 & $-1$ \\ \hline
 
				
				\end{tabular}}
	\end{table}
	
\end{columns}

\end{frame}

%---------------------------------------------------------

\begin{frame}
\frametitle{The Standard Model of Particle Physics}

\begin{columns}
	
	\column{0.60\textwidth}
	
	\begin{figure}
		\centering
		\includegraphics[width=\columnwidth]{./figures/SMcontentgaugebosonsfocus.png}
	\end{figure}
	
	\column{0.40\textwidth}
	{\small $G_{321} = SU(3)_C \tens \underbrace{SU(2)_L \tens U(1)_Y }_{\downarrow}   $}
	
	\hspace{1.735cm} {\small $SU(3)_C \tens U(1)_{EM}$}
	
	\vspace{1.0cm}
	\centering
	
	$SU(3)_C \rightarrow G_\mu^a$ \hspace{0.5cm} $a=1,..,8$ \\
	\vspace{0.5cm}
	
	
	
	$SU(2)_L \rightarrow W_\mu^i$ \hspace{0.5cm} $i=1, 2, 3$
	\vspace{0.5cm}
	
	
	
	$U(1)_Y \rightarrow B_\mu$
	
\end{columns}

\end{frame}

%---------------------------------------------------------

\begin{frame}
\frametitle{The Standard Model of Particle Physics}

\begin{columns}
	
	\column{0.50\textwidth}
	
	\begin{figure}
		\centering
		\includegraphics[width=\columnwidth]{./figures/SMcontenthiggsfocus.png}
	\end{figure}
	
	\column{0.50\textwidth}
	\centering
	{\small $G_{321} = SU(3)_C \tens \underbrace{SU(2)_L \tens U(1)_Y }_{\downarrow}   $}
	
	\hspace{1.735cm} {\small $SU(3)_C \tens U(1)_{EM}$}
	
			\begin{table}[]
		\centering
		{\tiny 	\begin{tabular}{c|c|c|c|c}
				&  $SU(3)_C \times SU(2)_L  \times U(1)_Y$ & $I_3$  & $Q_{\rm EM}$    \\ \hline \hline
				$Q=\begin{pmatrix} 
				u_{L}  \\
				d_{L} 
				\end{pmatrix}$	  & (\textbf{3}, \textbf{2}, $\frac{1}{3}$)  & $\begin{matrix} 
				1/2  \\
				-1/2 
				\end{matrix}$  & $\begin{matrix} 
				2/3  \\
				-1/3 
				\end{matrix}$      \\ 
				
				$u_R $ & $(\overline{\textbf{3}}, \textbf{1}, \frac{4}{3})$ & 0 & $2/3$ \\
				$d_R $ & $(\overline{\textbf{3}}, \textbf{1}, -\frac{2}{3})$ & 0 & $-1/3$ \\ \hline 
				
				
				$L=\begin{pmatrix} 
				\nu_{L}  \\
				e_{L} 
				\end{pmatrix}$	  & (\textbf{1}, \textbf{2}, $-1$)  & $\begin{matrix} 
				1/2  \\
				-1/2 
				\end{matrix}$  & $\begin{matrix} 
				0  \\
				-1 
				\end{matrix}$      \\ 
				
				$e_R $ & $(\overline{\textbf{1}}, \textbf{1}, -2)$ & 0 & $-1$ \\ \hline
				
			    	
				$\Phi=\begin{pmatrix} 
				\phi^+  \\
				\phi^{0}
				\end{pmatrix}$ & (\textbf{1}, \textbf{2}, $1$) &  $\begin{matrix} 
				1/2  \\
				-1/2 
				\end{matrix}$ & $\begin{matrix} 
				1  \\
				0 
				\end{matrix}$      \\ 
				
				
		\end{tabular}}
	\end{table}
	
\end{columns}

\end{frame}

%---------------------------------------------------------

\subsection{Problems of the Standard Model}

%---------------------------------------------------------

\begin{frame}
\frametitle{The Standard Model cannot be a complete theory!}

\vbox{
		\begin{minipage}[t][0.5\textheight][t]{\textwidth}
			\begin{columns}
				\begin{column}{0.5\textwidth}
					\color{red}
					\centering
					{\large  Gauge Hierarchy Problem!}
					\begin{figure}
						\centering
						\includegraphics[width=4.0cm]{./figures/gaugehierarchy.png}
					\end{figure}
				\end{column}
			    \pause
			    \begin{column}{0.5\textwidth}
			    	\color{red}
			    	\centering
			    	{\large  Dark Matter?}
			    	\begin{figure}
			    		\centering
			    		\includegraphics[width=4.7cm]{./figures/darkmatter.png}
			    	\end{figure}
			    \end{column}
			\end{columns}
		\end{minipage}
	\pause
	%Second half page
	
	
	    \nointerlineskip
	    \begin{minipage}[b][0.5\textheight][t]{\textwidth}
	    	\vspace{0.05in}
	    	\begin{columns}
	    		\begin{column}{0.5\textwidth}
	    			\color{red}
	    			\centering
	    			{\large  Neutrino Masses \& Oscillations!}
	    			\begin{figure}
	    				\centering
	    				\includegraphics[width=6.5cm]{./figures/neutrinomasses.png}
	    			\end{figure}
	    		\end{column}
    		    \pause
	    		\begin{column}{0.5\textwidth}
	    			\color{red}
	    			\centering
	    			{\large  Why $SU(3)_C \tens SU(2)_L  \tens U(1)_Y$?}
	    			\begin{figure}
	    				\centering
	    				\includegraphics[width=3.8cm]{./figures/Unification.png}
	    			\end{figure}
	    		\end{column}
	    	\end{columns}
    	\end{minipage}
	}
	
\end{frame}

%---------------------------------------------------------










%---------------------------------------------------------

\subsection{Minimal Supersymmetric Standard Model (MSSM)}

%---------------------------------------------------------

\begin{frame}
\frametitle{Minimal Supersymmetric Standard Model (MSSM)}

\vbox{
	\begin{minipage}[t][0.5\textheight][t]{\textwidth}
		\begin{columns}
\invisible{
			\begin{column}{0.4\textwidth}
				\color{red}
				\centering
				\begin{figure}
					\includegraphics[scale=0.3]{./figures/susyparticles.png}
				\end{figure}
			\end{column}
            }
			\begin{column}{0.6\textwidth}
				\color{black}
				\centering
				\begin{figure}
					\includegraphics[scale=0.3]{./figures/susyparticles.png}
				\end{figure}
			\end{column}
		\end{columns}
	\end{minipage}
	%Second half page
	
\nointerlineskip
\begin{minipage}[b][0.5\textheight][t]{\textwidth}
	\vspace{0.05in}
	\begin{columns}
		\begin{column}{0.5\textwidth}
			\color{black}
			\centering
		\end{column}
		\begin{column}{0.5\textwidth}
			\color{red}
			\centering
		\end{column}
	\end{columns}
\end{minipage}
}
\end{frame}

%---------------------------------------------------------













%---------------------------------------------------------

\begin{frame}
\frametitle{Minimal Supersymmetric Standard Model (MSSM)}

\vbox{
	\begin{minipage}[t][0.5\textheight][t]{\textwidth}
		\begin{columns}
			\begin{column}{0.5\textwidth}
				\color{black}
				\centering
				\vspace{-0.5cm}
				\begin{table}[]
					\centering
					{\tiny 	\begin{tabular}{c|c|c|c|c}
							&  $SU(3)_C \times SU(2)_L  \times U(1)_Y$ & $I_3$  & $Q_{\rm EM}$    \\ \hline \hline
							$Q=\begin{pmatrix} 
							u_{L}  \\
							d_{L} 
							\end{pmatrix}$	  & (\textbf{3}, \textbf{2}, $\frac{1}{3}$)  & $\begin{matrix} 
							1/2  \\
							-1/2 
							\end{matrix}$  & $\begin{matrix} 
							2/3  \\
							-1/3 
							\end{matrix}$      \\ 
							
							$u_R $ & $(\overline{\textbf{3}}, \textbf{1}, \frac{4}{3})$ & 0 & $2/3$ \\
							$d_R $ & $(\overline{\textbf{3}}, \textbf{1}, -\frac{2}{3})$ & 0 & $-1/3$ \\ \hline 
							
							
							$L=\begin{pmatrix} 
							\nu_{L}  \\
							e_{L} 
							\end{pmatrix}$	  & (\textbf{1}, \textbf{2}, $-1$)  & $\begin{matrix} 
							1/2  \\
							-1/2 
							\end{matrix}$  & $\begin{matrix} 
							0  \\
							-1 
							\end{matrix}$      \\ 
							
							$e_R $ & $(\overline{\textbf{1}}, \textbf{1}, -2)$ & 0 & $-1$ \\ \hline \hline
							
							
							
							$H_u=\begin{pmatrix} 
							H_u^+  \\
							H_u^0 
							\end{pmatrix}$	  & (\textbf{1}, \textbf{2}, $1$)  & $\begin{matrix} 
							1/2  \\
							-1/2 
							\end{matrix}$  & $\begin{matrix} 
							1  \\
							0 
							\end{matrix}$  \\ \hline
							
							$H_d=\begin{pmatrix} 
							H_d^0  \\
							H_d^- 
							\end{pmatrix}$	  & (\textbf{1}, $\overline{\textbf{2}}$, $1$)  & $\begin{matrix} 
							1/2  \\
							-1/2 
							\end{matrix}$  & $\begin{matrix} 
							1  \\
							0 
							\end{matrix}$  \\ \hline
							
					\end{tabular}}
				\end{table}
			\end{column}
			\begin{column}{0.5\textwidth}
				\color{black}
				\centering
				\vspace{-1.0cm}
				\begin{figure}
					\includegraphics[scale=0.25]{./figures/susyparticles.png}
				\end{figure}
			\end{column}
		\end{columns}
	\end{minipage}
	%Second half page
	
	\pause
	
	\nointerlineskip
	
	\begin{minipage}[b][0.5\textheight][t]{\textwidth}
		\vspace{0.05in}
		\begin{columns}
			\begin{column}{0.5\textwidth}
				\color{black}
				\centering
				\vspace{-0.5cm}
	            {\small  \begin{equation*}
					W=\mu H_{u}H_{d}+Y_{u}^{ij}Q_{i}H_{u}u^{c}_{j}-Y_{d}^{ij}Q_{i}H_{d}d^{c}_{j}-Y_{e}^{ij}L_{i}H_{d}e^{c}_{j} 
					\end{equation*}}
				\pause
				\centering
				\vspace{-0.45cm}
				\begin{alertblock}{Solutions to the SM problems:}
					\begin{itemize}
						\item No Gauge Hierarchy Problem!
						\item Dark Matter Candidate
						\item Gauge Coupling Unification
				    \end{itemize}
				\end{alertblock}
			\end{column}
			\pause
			\begin{column}{0.5\textwidth}
				\color{red}
				\centering
				\begin{alertblock}{ But still..}
					 \begin{itemize}
					 	\centering
						\item Neutrino mass ?
						\item $\mu$ Problem
						\item MSSM requires substantial fine-tuning
					\end{itemize}
				\end{alertblock}
				
			\end{column}
		\end{columns}
	\end{minipage}
}

\end{frame}
%---------------------------------------------------------





%---------------------------------------------------------

\section{U(1)$_{B-L} \times$ U(1)$_{R}$ Extended MSSM}
\subsection{Model Building}

%---------------------------------------------------------

\begin{frame}
\frametitle{Supersymmetric U(1)$_{B-L} \times$ U(1)$_{R}$ Model (BLRinvSeesaw)}

\begin{columns}
\begin{column}{0.5\textwidth}
	\centering
	{\small GUT-inspired U(1)$_{B-L} \times$ U(1)$_{R}$ extended \\
		MSSM symmetry breaking scheme}

{\scriptsize 		$$SO(10) \to SU(3)_C \times SU(2)_L \times SU(2)_R \times U(1)_{B-L}$$ \\
	\vspace{-0.7cm}
	$$\hspace{0.8cm}\to SU(3)_C \times SU(2)_L \times U(1)_R \times U(1)_{B-L} $$ \\
    \vspace{-0.7cm}
	$$\hspace{-0.5cm}\to SU(3)_C \times SU(2)_L \times  U(1)_{Y} $$}

\pause


$W= W_{MSSM} +Y_{\nu}^{ij}L_{i}H_{u}N^{c}_{i}+ Y^{ij}_{s}N^{c}_{i}\singR S - \mu_{R}\singRbar \singR + \mu_{S}S S $


\end{column}

\begin{column}{0.5\textwidth}
	\color{red}
	\centering
	
\begin{figure}
	\includegraphics[scale=0.40]{./figures/BLRinvSeesawfields.png}
\end{figure}
\pause
	\vspace{-0.5cm}
	\begin{alertblock}{ Motivation}
		\begin{itemize}
			\centering
			\item  Neutrino mass problem $\rightarrow$ \underline{Solved}!
			\item Extra DM candidate
			\item Better resolution to muon g-2
			\item Relatively light Higgs boson masses
		\end{itemize}
	\end{alertblock}
	
\end{column}
\end{columns}	

\end{frame}





%---------------------------------------------------------

\subsection{Parameter Space \& Constraints}

%---------------------------------------------------------

\begin{frame}
\frametitle{Parameter Space of BLRinvSeesaw}
\color{red}
\centering
Universal Boundary Conditions
\pause
\begin{table}
	\color{black}
	\setlength\tabcolsep{7pt}
	\renewcommand{\arraystretch}{1.4}
	\begin{tabular}{c|c||c|c}
		Parameter      & Scanned range& Parameter      & Scanned range\\
		\hline
		$m_0$          & $[0., 3.]$~TeV   & $v_{R}$               & $[6.5, 20.]$~TeV\\
		$M_{1/2}$      & $[0., 3.]$~TeV   & $diag(Y_{\nu}^{ij})$  & $[0.001, 0.99]$\\
		$A_0/m_0$      & $[-3., 3.]$      & $diag(Y_{s}^{ij})$    & $[0.001, 0.99]$\\
		$\tan\beta$    & $[0., 60.]$      & {\rm sign of} $\mu$   & {\rm positive} \\
		$\tan\beta_R$  & $[1., 1.2]$    & {\rm sign of} $\mu_R$ & {\rm positive or negative} \\ 
	\end{tabular}
	\end{table}
Scanned parameter space
\end{frame}

%---------------------------------------------------------






%---------------------------------------------------------

\begin{frame}
\frametitle{Experimental Constraints}

{\small \begin{table}{
		\setlength\tabcolsep{7pt}
		\renewcommand{\arraystretch}{1.6}
		\begin{tabular}{l|c|c|l|c|c}
			Observable & Constraints &  & Observable & Constraints & \\
			\hline
			$m_{h_1} $ & $ [122,128] $ GeV                                &  &
			$m_{\widetilde{t}_1} $                                 & $ \geqslant 730 $ GeV & \\
			$m_{\widetilde{g}} $                                     & $ > 1.75 $ TeV & &
			$ m_{\chi_1^\pm} $                                    & $ \geqslant 103.5 $ GeV &  \\
			$m_{\widetilde{\tau}_1} $                                & $ \geqslant 105 $ GeV &  & 
			$m_{\widetilde{b}_1} $                                 & $ \geqslant 222 $ GeV & \\
			$m_{\widetilde{q}} $                                     & $ \geqslant 1400 $ GeV & &
			$m_{\widetilde{\tau}_1} $                              & $ > 81 $ GeV &  \\
			$m_{\widetilde{e}_1} $                                   & $ > 107 $ GeV &  &
			$m_{\widetilde{\mu}_1} $                               & $ > 94 $ GeV &  \\
			$\chi^2(\hat{\mu})$                                    & $\leq 2.
			3 $ &  &
			BR$(B^0_s \to \mu^+\mu^-) $ & $[1.1,6.4] \times 10^{-9}$  &
			 \\ 
			$\displaystyle  \frac{{\rm BR}(B \to \tau\nu_\tau)}
			{{\rm BR}_{SM}(B \to \tau\nu_\tau)} $ & $ [0.15,2.41] $ &
			 &
			BR$(B^0 \to X_s \gamma) $ & $  [2.99,3.87]\times10^{-4} $ &
			\\
			$m_{Z^{\prime}} $                                        & $ > 3.5 $ TeV & &
			$\Omega_{DM}h^{2} $                                            &  [0.09-0.14] &                       \\  
		\end{tabular}
		 }
\end{table}}

\end{frame}






%--------------------------------------------------------

\section{Results}
\subsection{$\widetilde{\chi}_1^0$ DM scenario}

%--------------------------------------------------------

\begin{frame}
\frametitle{Case I: Neutralino $\widetilde{\chi}_1^0$ Dark Matter Scenario}
\centering
\includegraphics[scale=0.35]{./figures/neutralino_relic_DMcontent.png} 
\pause
\includegraphics[scale=0.35]{./figures/nucleoncross_mchi1_color.png} \\
\end{frame}

%--------------------------------------------------------



%--------------------------------------------------------

\begin{frame}
\frametitle{Case I: Neutralino $\widetilde{\chi}_1^0$ Dark Matter Scenario}
\centering
Funnel Channels $\rightarrow m_{A_1}, m_{h_3}$
\includegraphics[scale=0.35]{./figures/mchi1_massAh3.png} 
\includegraphics[scale=0.35]{./figures/massAh3_massh3.png}

\graybullet Excluded solutions \\
\bluebullet Solutions consistent with all constraints except for the relic density bound \\
\redbullet Solutions consistent with all constraints including the relic density bound

\end{frame}

%--------------------------------------------------------




%--------------------------------------------------------
\subsection{Heavy Z boson}
%--------------------------------------------------------

\begin{frame}
\frametitle{$Z^\prime$ mass limit}
\centering
\includegraphics[scale=0.35]{./figures/totxsectionZpll.png} 
\pause
\includegraphics[scale=0.35]{./figures/totxsectionZpqqbar.png} \\
\pause
$M_{Z^\prime} > 3.5$ TeV
\end{frame}

%--------------------------------------------------------









%--------------------------------------------------------

\subsection{$\widetilde{\nu}_1$ DM  scenario}

%--------------------------------------------------------

\begin{frame}
\frametitle{Case II: Sneutrino $\widetilde{\nu}_1$ Dark Matter Scenario}
\centering
\includegraphics[scale=0.35]{./figures/relic_sneutrino_rightcontent.png} 
\includegraphics[scale=0.35]{./figures/relic_sneutrino_Scontent.png} \\

Only 16 solutions out of 100,000 total solutions are consistent \\
with the relic density bound.

\end{frame}

%--------------------------------------------------------





%--------------------------------------------------------

\begin{frame}
\frametitle{The Effect of $Z^{\prime}$ mass in $\widetilde{\nu}_1$ DM Scenario}
\centering
\includegraphics[scale=0.35]{./figures/protoncross_massSv1.png} 
\includegraphics[scale=0.35]{./figures/neutroncross_massSv1.png} \\

\invisible{	
\centering
\includegraphics[scale=0.35]{./figures/protoncross_massSv1foroldZp.png} 
\includegraphics[scale=0.35]{./figures/neutroncross_massSv1foroldZp.png} \\	
	}

\end{frame}

%--------------------------------------------------------


%--------------------------------------------------------

\begin{frame}
\frametitle{The Effect of $Z^{\prime}$ mass in $\widetilde{\nu}_1$ DM Scenario}
\centering
\includegraphics[scale=0.25]{./figures/protoncross_massSv1.png} 
\includegraphics[scale=0.25]{./figures/neutroncross_massSv1.png} \\

	
	\centering
	\includegraphics[scale=0.28]{./figures/protoncross_massSv1foroldZp.png} 
	\includegraphics[scale=0.28]{./figures/neutroncross_massSv1foroldZp.png} \\	


\end{frame}

%--------------------------------------------------------






%--------------------------------------------------------

\subsection{Muon Anomalous Magnetic Moment}

%--------------------------------------------------------

\begin{frame}
\frametitle{Muon Anomalous Magnetic Moment}
\centering

\begin{columns}
	\begin{column}{0.25\textwidth}
{\scriptsize 	
		$\Delta{a_\mu} = a_\mu^{\rm exp} - a_\mu^{\rm SM}$ \\
$ \hspace{0.65cm} = 28.7 \times 10^{-10} $
}
\pause

\vspace{0.5cm}
\includegraphics[scale=0.25]{./figures/muong2diagram1.png}
\vspace{0.5cm}
\includegraphics[scale=0.25]{./figures/muong2diagram2.png}
\pause

	\end{column}

	\begin{column}{0.75\textwidth}
		
		\includegraphics[scale=0.25]{./figures/DAMU_m0_BOTH.png} 
		\includegraphics[scale=0.25]{./figures/DAMU_mhf_BOTH.png} \\
		
		
		\centering
		\includegraphics[scale=0.25]{./figures/DAMU_mu_BOTH.png} 
		\pause
		\includegraphics[scale=0.25]{./figures/DAMU_tanb_BOTH.png} \\	
		
	\end{column}
		
	
\end{columns}


\end{frame}

%--------------------------------------------------------
















%---------------------------------------------------------

\section{Conclusion and Future Studies}

\subsection{New results}
%---------------------------------------------------------


\begin{frame}
\frametitle{New results based on the non-universality in $\singR$ masses}

\color{red}
\centering
Tadpole equations are solved in 
$(\mu, B_\mu, m_{\singRbar}^2, m_{\singR}^2 )$ basis \\
$m_{\singRbar}^2 \neq m_{\singR}^2 \neq m_0^2$ at $M_{\rm GUT}$ 
\pause
\begin{table}
	\color{black}
	\setlength\tabcolsep{7pt}
	\renewcommand{\arraystretch}{1.4}
	\begin{tabular}{c|c||c|c}
		Parameter      & Scanned range& Parameter      & Scanned range\\
		\hline
		$m_0$          & $[0., 3.]$~TeV   & $v_{R}$               & $[6.5, 20.]$~TeV\\
		$M_{1/2}$      & $[0., 3.]$~TeV   & $diag(Y_{\nu}^{ij})$  & $[0.001, 0.99]$\\
		$A_0/m_0$      & $[-3., 3.]$      & $diag(Y_{s}^{ij})$    & $[0.001, 0.99]$\\
		$\tan\beta$    & $[0., 60.]$      & {\rm sign of} $\mu$   & {\rm positive} \\
		$\tan\beta_R$  & $[1., 1.2]$    & \color{red} $\mu_R$ &  \color{red} $[-4.2, 6.]$~TeV \\ 
		  &   & \color{red} $\Delta m_{\singR}^2$ &  \color{red} $[0, 10.]$~TeV \\ 
	\end{tabular}
\end{table}

\color{red} where $\Delta m_{\singR}^2 = m_{\singRbar}^2 - m_{\singR}^2 $ 

\end{frame}



\begin{frame}
\frametitle{$\widetilde{\chi}_1^0$ DM based on the non-universality in $\singR$ masses}

\centering
\includegraphics[scale=0.35]{./figures/nonBL_nucleoncross_mchi1_new.png} 
\includegraphics[scale=0.35]{./figures/nonBL_nucleonSIstrength_mchi1.png} \\

where $\sigma_{SI}^{nucleon}$ is rescaled as $\widetilde{\sigma}_{SI}^{nucleon} = \sigma_{SI}^{nucleon} \frac{\Omega h^2}{\Omega_{DM}^{Planck}h^2}$

\end{frame}



\begin{frame}
\frametitle{$\widetilde{\nu}_1$ DM based on the non-universality in $\singR$ masses}

\centering
\includegraphics[scale=0.35]{./figures/nonBL_relic_sneutrino_rightcontent.png} 
\includegraphics[scale=0.35]{./figures/nonBL_relic_sneutrino_scalarcontent.png} \\

Sneutrino DM solutions can be also obtained with $M_{Z^{\prime}} > 3.5 $ TeV bound.
\end{frame}



\begin{frame}
\frametitle{}
\centering
{\Huge Thank you!}

\end{frame}






\end{document}